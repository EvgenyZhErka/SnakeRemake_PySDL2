\documentclass[14pt, oneside]{altsu-report}

\worktype{Отчёт по практике на тему:}
\title{Игра "Snake Classic"}
\author{Е.\,Д.~Яковенко}
\groupnumber{5.205-2}
\GradebookNumber{1337}
\supervisor{И.\,А.~Шмаков}
\supervisordegree{к.ф.-м.н., доцент}
\ministry{Министерство науки и высшего образования}
\country{Российской Федерации}
\fulluniversityname{ФГБОУ ВО Алтайский государственный университет}
\institute{Институт цифровых технологий, электроники и физики}
\department{Кафедра вычислительной техники и электроники}
\departmentchief{В.\,В.~Пашнев}
\departmentchiefdegree{к.ф.-м.н., доцент}
\shortdepartment{ВТиЭ}

\date{\the\year}

% Подключение файлов с библиотекой.
\addbibresource{graduate-students.bib}

% Пакет для отладки отступов.
%\usepackage{showframe}

\begin{document}
\maketitle

\setcounter{page}{2}
\tableofcontents

\chapter*{Введение}
\phantomsection\addcontentsline{toc}{chapter}{ВВЕДЕНИЕ}

\textbf{Актуальность:}
Игра, которая является клоном классической змейки, остается актуальной в наше время по ряду причин. Она обладает простым, но захватывающим геймплеем, который привлекает как детей, так и взрослых. Благодаря своей низкой сложности и затягивающей динамике, данная игра подходит для развлечения и отдыха, а также может быть использована для улучшения логического мышления, координации движений и развития стратегического мышления у игроков.

Создание и разработка подобной игры на современных технологиях представляет собой отличную возможность для программистов и разработчиков углубить свои знания и навыки в области геймдева. Работа над игрой "Змейка" может помочь им освоить новые инструменты и технологии, а также получить практический опыт в создании игр для различных платформ и устройств.

Образовательные цели создания игры "Змейка" могут включать в себя как углубление знаний в области программирования и разработки игр, так и развитие креативности и аналитических способностей. Работа над проектом может помочь разработчикам улучшить свои навыки в написании кода, оптимизации игровых механик и взаимодействия с пользователем, а также научиться работать в команде и решать проблемы, возникающие в процессе разработки.

Таким образом, игра, основанная на концепции змейки, представляет собой актуальный и интересный проект для разработки, который может быть полезным как для личного развития, так и для профессионального роста в области геймдева. Создание подобных игр не только развлекает и увлекает, но и способствует приобретению новых знаний и навыков в области программирования и разработки игр.

\textbf{Цель:}
Цели разработки игры "Змейка" на Python с использованием библиотеки PySDL2 могут быть разносторонними:
\begin{enumerate}
    \item Обучение и образовательные цели: Разработка игры "Змейка" на языке Python может быть направлена на обучение студентов, начинающих программистов или любых людей, интересующихся программированием. Это может помочь им освоить основы создания игр и использование графических библиотек, таких как PySDL2, что может способствовать в их профессиональном и личном развитии.
    \item Практический опыт: Разработка игры "Змейка" на Python с использованием PySDL2 может служить для получения практического опыта в разработке игр и работы с графикой. Это может быть полезно для студентов и начинающих разработчиков, позволяя им применить свои знания в реальном проекте.
    \item Демонстрация возможностей библиотеки: Разработка может быть использована для демонстрации возможностей библиотеки PySDL2, показывая, как можно создавать игры с использованием этой технологии. Это может заинтересовать других разработчиков и способствовать популяризации библиотеки.
    \item Развлечение и создание качественного продукта: Игра "Змейка" на Python может быть разработана с целью создания развлекательного продукта, который привлечет игроков своим простым, но увлекательным геймплеем, создавая при этом качественный продукт развлечения.
\end{enumerate}
В целом, разработка игры "Змейка" на Python с использованием библиотеки PySDL2 может быть ориентирована на образовательные цели, получение практического опыта, демонстрацию возможностей технологии и создание развлекательного продукта, тем самым удовлетворяя различные потребности и интересы в различных областях разработки и использования программного обеспечения.

\textbf{Задачи:}
\begin{enumerate}
\item Реализовать игровые элементы:
\begin{enumerate}
\item Игровое поле
\item Змейку
\item Еду
\item Генериуемые препятствия
\end{enumerate}
\item Игровой процесс:
\begin{enumerate}
\item Обработка случая, когда змейка съедает еду и увеличивает свою длину.
\item Усложнение игры с увеличением скорости перемещения змейки.
\item Реализация движения змейки по полю в зависимости от указанных пользователем клавиш.
\item Реализация функции окончания игры при достижении змейкой стенки поля или произошедшем столкновении с самой собой.
\item Реализовать управление змейкой с помощью клавиатуры.
\end{enumerate}
\item Меню
\begin{enumerate}
\item Кнопка "Играть"
\item Кнопка "Автор"
\item Кнопка "Выход"
\end{enumerate}
\end{enumerate}

% Подключение первой главы (теория):
\chapter{\label{ch:ch01}ГЛАВА 1} % Нужно сделать главу в содержании заглавными буквами

\section{\label{sec:ch01/sec01}Раздел 1: история библиотеки PySDL2}

\subsection{\label{subsec:ch01/sec01/sub01}Подраздел 1: пример нумерованного списка}

Пример <<ковычек>> и тире ---.

Пример нумерованного списка:
\begin{enumerate}
\item Первый элемент.
\item Второй элемент.
\end{enumerate}

\subsection{\label{subsec:ch01/sec01/sub02}Подраздел 2: пример маркерованного списка}

Пример маркерованного списка:
\begin{itemize}
\item первый элемент;
\item второй элемент.
\end{itemize}

\section{\label{sec:ch01/sec02}Раздел 2: много уровневые списки}

\subsection{\label{subsec:ch01/sec02/sub01}Подраздел 1: пример нумерованного списка}

Пример вложенного нумерованного списка:
\begin{enumerate}
\item Первый элемент:
\begin{enumerate}
\item Первый элемент первого элемента;
\item Второй элемент первого элемента;
\end{enumerate}
\item Второй элемент:
\begin{enumerate}
\item Первый элемент второго элемента;
\item Второй элемент второго элемента.
\end{enumerate}
\end{enumerate}

\subsection{\label{subsec:ch01/sec02/sub02}Подраздел 2: пример маркерованного списка}

Пример вложенного маркерованного списка:
\begin{itemize}
\item первый элемент:
\begin{itemize}
\item первый элемент первого элемента;
\item второй элемент первого элемента;
\end{itemize}
\item Второй элемент:
\begin{itemize}
\item первый элемент второго элемента;
\item второй элемент второго элемента.
\end{itemize}
\end{itemize}

Пример ссылки на рисунок в документе~\ref{fig:example01}.
\begin{figure}[h]
    \centering
    \includegraphics[width=0.5\textwidth]{./images/fibonacci.png}
    \caption{\centering\label{fig:example01}Пример рисунка в формате PNG.}
\end{figure}

Пример ссылки на рисунок в документе~\ref{fig:example02}.
\begin{figure}[h]
    \centering
    \includesvg[width=0.5\textwidth]{./images/fibonacci.svg}
    \caption{\centering\label{fig:example02}Пример рисунка в формате SVG.}
\end{figure}

Пример ссылки на таблицу в документе~\ref{tab:example01}.
\begin{table}[H]
\caption{\centering\label{tab:example01}Системные требования}
\begin{tabular}{|p{3 cm}|p{3 cm}|p{3 cm}|p{5 cm}|}
\hline
Минимальные требования & 1 & 2 & 3 \\ \hline
Версия операционной системы & 1 & 2 & 3 \\ \hline
Процессор & 1 & 2 & 3 \\ \hline
Графический API & 1 & 2 & 3 \\ \hline
\end{tabular}
\end{table}

Пример ссылки на таблицу в документе~\ref{tab:example02}.
\begin{table}[H]
\caption{\centering\label{tab:example02}Системные требования}
\begin{tabular}{|p{3 cm}|p{3 cm}|p{3 cm}|p{5 cm}|}
\hline
Минимальные требования & 1 & 2 & 3 \\ \hline
Версия операционной системы & 1 & 2 & 3 \\ \hline
Процессор & 1 & 2 & 3 \\ \hline
Графический API & 1 & 2 & 3 \\ \hline
\end{tabular}
\end{table}

Пример использования minted для оформления кода и ссылка на этот код~\ref{code:fibonacci}.
\begin{code}
\captionof{listing}{\centering\label{code:fibonacci}Пример программы вычисления n-ой последовательности Фибоначчи}
\vspace{-\baselineskip}\inputminted{python}{src/fibonacci.py}
\end{code}

Пример использования minted для оформления кода и ссылка на этот код~\ref{code:example02}.
\begin{code}
\captionof{listing}{\centering\label{code:example02}Сложение двух массивов параллельно десятью потоками (пример из https://ru.wikipedia.org/wiki/OpenMP)}
\vspace{-\baselineskip}\begin{minted}{C}
#include <stdio.h>
#include <omp.h>
#define N 100

int main(int argc, char *argv[]) {
  double a[N], b[N], c[N];
  int i;
  omp_set_dynamic(0); // запретить библиотеке openmp менять число потоков во время исполнения
  omp_set_num_threads(10); // установить число потоков в 10
  // инициализируем массивы
  for (i = 0; i < N; i++) {
      a[i] = i * 1.0;
      b[i] = i * 2.0;
  }
  // вычисляем сумму массивов
#pragma omp parallel for shared(a, b, c) private(i)
   for (i = 0; i < N; i++)
     c[i] = a[i] + b[i];

  printf ("%f\n", c[10]);
  return 0;
}
\end{minted}
\end{code}

% Подключение второй главы (практическая часть):
\chapter{\label{ch:ch02}ГЛАВА 2}

\section{\label{sec:ch02/sec01}Раздел 1}

\subsection{\label{subsec:ch02/sec01/sub01}Подраздел 1}

Пример ссылки на рисунок в документе~\ref{fig:example03}.
\begin{figure}[h]
    \centering
    \includegraphics[width=0.5\textwidth]{./images/fibonacci.png}
    \caption{\centering\label{fig:example03}Пример рисунка в формате PNG.}
\end{figure}

Пример ссылки на рисунок в документе~\ref{fig:example04}.
\begin{figure}[h]
    \centering
    \includesvg[width=0.5\textwidth]{./images/fibonacci.svg}
    \caption{\centering\label{fig:example04}Пример рисунка в формате SVG.}
\end{figure}

\subsection{\label{subsec:ch02/sec01/sub02}Подраздел 2}

\section{\label{sec:ch02/sec02}Раздел 2}

\subsection{\label{subsec:ch02/sec02/sub01}Подраздел 1}

\subsection{\label{subsec:ch02/sec02/sub02}Подраздел 2}


% Подключение третий главы (практическая часть с тестированием:
\chapter{\label{ch:ch03}ГЛАВА 3}

Пример ссылок:
\begin{enumerate}
\item на главу~\ref{ch:ch01};
\item на раздел~\ref{sec:ch01/sec01} главы~\ref{ch:ch01};
\item на раздел~\ref{sec:ch02/sec01} главы~\ref{ch:ch02};
\item на приложение на странице~\pageref{appendix1};
\item на код на странице~\pageref{code:pi-example}.
\end{enumerate}

\section{\label{sec:ch03/sec01}Раздел 1}

\subsection{\label{subsec:ch03/sec01/sub01}Подраздел 1}

\subsection{\label{subsec:ch03/sec01/sub02}Подраздел 2}

\section{\label{sec:ch03/sec02}Раздел 2}

\subsection{\label{subsec:ch03/sec02/sub01}Подраздел 1}

\subsection{\label{subsec:ch03/sec02/sub02}Подраздел 2}

Пример ссылки на рисунок в документе~\ref{fig:example05}.
\begin{figure}[h]
    \centering
    \includegraphics[width=0.5\textwidth]{./images/fibonacci.png}
    \caption{\centering\label{fig:example05}Пример рисунка в формате PNG.}
\end{figure}

Пример ссылки на рисунок в документе~\ref{fig:example06}.
\begin{figure}[h]
    \centering
    \includesvg[width=0.5\textwidth]{./images/fibonacci.svg}
    \caption{\centering\label{fig:example06}Пример рисунка в формате SVG.}
\end{figure}

Пример ссылки на таблицу в документе~\ref{tab:example05}.
\begin{table}[H]
\caption{\centering\label{tab:example05}Системные требования}
\begin{tabular}{|p{3 cm}|p{3 cm}|p{3 cm}|p{5 cm}|}
\hline
Минимальные требования & 1 & 2 & 3 \\ \hline
Версия операционной системы & 1 & 2 & 3 \\ \hline
Процессор & 1 & 2 & 3 \\ \hline
Графический API & 1 & 2 & 3 \\ \hline
\end{tabular}
\end{table}

Пример ссылки на таблицу в документе~\ref{tab:example06}.
\begin{table}[H]
\caption{\centering\label{tab:example06}Системные требования}
\begin{tabular}{|p{3 cm}|p{3 cm}|p{3 cm}|p{5 cm}|}
\hline
Минимальные требования & 1 & 2 & 3 \\ \hline
Версия операционной системы & 1 & 2 & 3 \\ \hline
Процессор & 1 & 2 & 3 \\ \hline
Графический API & 1 & 2 & 3 \\ \hline
\end{tabular}
\end{table}

Пример использования minted для оформления кода и ссылка на этот код~\ref{code:example05}.
\begin{code}
\captionof{listing}{\centering\label{code:example05}Вычисление последовательности Фибоначчи}
\vspace{-\baselineskip}\begin{minted}{C}
#include <stdio.h>
#include <omp.h>
#define N 100

int main(int argc, char *argv[]) {
  double a[N], b[N], c[N];
  int i;
  omp_set_dynamic(0); // запретить библиотеке openmp менять число потоков во время исполнения
  omp_set_num_threads(10); // установить число потоков в 10
  // инициализируем массивы
  for (i = 0; i < N; i++) {
      a[i] = i * 1.0;
      b[i] = i * 2.0;
  }
  // вычисляем сумму массивов
#pragma omp parallel for shared(a, b, c) private(i)
   for (i = 0; i < N; i++)
     c[i] = a[i] + b[i];

  printf ("%f\n", c[10]);
  return 0;
}
\end{minted}
\end{code}

код игры в разработке~\ref{code:example06}.
\begin{code}
\captionof{listing}{\centering\label{code:example06}
\vspace{-\baselineskip}\begin{minted}{Python}
#include <stdio.h>
#include <omp.h>
#define N 100

int main(int argc, char *argv[]) {
  double a[N], b[N], c[N];
  int i;
  omp_set_dynamic(0); // запретить библиотеке openmp менять число потоков во время исполнения
  omp_set_num_threads(10); // установить число потоков в 10
  // инициализируем массивы
  for (i = 0; i < N; i++) {
      a[i] = i * 1.0;
      b[i] = i * 2.0;
  }
  // вычисляем сумму массивов
#pragma omp parallel for shared(a, b, c) private(i)
   for (i = 0; i < N; i++)
     c[i] = a[i] + b[i];

  printf ("%f\n", c[10]);
  return 0;
}
\end{minted}
\end{code}


\chapter*{Заключение}
\phantomsection\addcontentsline{toc}{chapter}{ЗАКЛЮЧЕНИЕ}
В рамках данной курсовой работы была разработана игра "змейка" на языке программирования Python с использованием библиотеки PySDL2. Работа позволила изучить основы разработки игр, такие как управление игровым персонажем, обработка столкновений, генерация случайных объектов и подсчет очков. Реализация проекта не только позволила улучшить навыки программирования на Python, но и ознакомила с различными аспектами создания игрового приложения. Всего за курс работы был приобретен ценный опыт, добавивший новые знания в области разработки компьютерных игр и программирования.
\begin{enumerate}
\item Пример ссылки на электронный источник~\cite{wikiRUBitbucket,wikiRUIdSoftware,wikiRUGitHub}.
\item Пример ссылки на книгу одного автора~\cite{book1author}.
\item Пример ссылки на книгу 5-ти и более авторов~\cite{book5author}.
\end{enumerate}

\newpage
\phantomsection\addcontentsline{toc}{chapter}{СПИСОК ИСПОЛЬЗОВАННОЙ ЛИТЕРАТУРЫ}
\printbibliography[title={Список использованной литературы}]

\appendix
\newpage
\chapter*{\label{appendix1}Приложение}
\phantomsection\addcontentsline{toc}{chapter}{ПРИЛОЖЕНИЕ}
%\section*{\centering\label{code:appendix}Текст программы}

\begin{center}
\label{code:appendix}Текст программы
\end{center}

\begin{code}
\captionof*{listing}{\centering\label{code:pi-example}Пример программы вычисления числа $\pi$ на языке \textit{C} с использованием \textit{MPI} (пример из https://ru.wikipedia
.org/wiki/Message\_Passing\_Interface)}
\vspace{-1cm}\inputminted{C}{src/pi-mpi.c}
\end{code}

\end{document}
